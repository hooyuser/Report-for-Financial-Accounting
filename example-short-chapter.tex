%%%% Time-stamp: <2012-08-20 17:41:39 vk>

%% example text content
%% scrartcl and scrreprt starts with section, subsection, subsubsection, ...
%% scrbook starts with part (optional), chapter, section, ...

\chapter{Taxes}

The amounts of income tax benefits or expenses for the year ended December 31,2015 and the year ended December 31,2016 are tabulated as follows.

\begin{center}
\begin{tabular}{lcc}
	\toprule
	Fiscal year&2015&2016\\
	\midrule
	Income tax benefits/(expenses)&14,262&(179,500)\\
	\bottomrule
\end{tabular}
\end{center}


Current income taxes are provided on the basis of net income for financial reporting purposes, adjusted for income and expense items which are not assessable or deductible for income tax purposes, in accordance with the regulations of the relevant tax jurisdictions. The Group follows the liability method of accounting for income taxes. Under this method, deferred tax assets and liabilities are determined based on the temporary differences between the financial statements carrying amounts and tax bases of existing assets and liabilities by applying enacted statutory tax rates that will be in effect in the period in which the temporary differences are expected to reverse. The Group records a valuation allowance to reduce the amount of deferred tax assets if based on the weight of available evidence, it is more-likely-than-not that some portion, or all, of the deferred tax assets will not be realized. The effect on deferred taxes of a change in tax rates is recognized in the Consolidated Statements of Operations and Comprehensive Loss in the period of change. The Group recognizes in its consolidated financial statements the benefit of a tax position if the tax position is “more likely than not” to prevail based
on the facts and technical merits of the position. Tax positions that meet the “more likely than not” recognition threshold are measured at the largest amount
of tax benefit that has a greater than fifty percent likelihood of being realized upon settlement. The Group estimates its liability for unrecognized tax benefits
which are periodically assessed and may be affected by changing interpretations of laws, rulings by tax authorities, changes and/or developments with
respect to tax audits, and expiration of the statute of limitations. The ultimate outcome for a particular tax position may not be determined with certainty
prior to the conclusion of a tax audit and, in some cases, appeal or litigation process. The actual benefits ultimately realized may differ from the Group’s
estimates. As each audit is concluded, adjustments, if any, are recorded in the Group’s consolidated financial statements in the period in which the audit is
concluded. Additionally, in future periods, changes in facts, circumstances and new information may require the Group to adjust the recognition and
measurement estimates with regard to individual tax positions. Changes in recognition and measurement estimates are recognized in the period in which the
changes occur. As of December 31, 2014, 2015 and 2016, the Group did not have any significant unrecognized uncertain tax positions.

\chapter{Dividends}
\section{Factors on Dividend Policy}
There are several factors that influence whether a company pays a dividend and how much it chooses to pay. Some of the most influential factors are listed as follows.
\begin{itemize}
	\item \textbf{Income stability.} 
	Income stability is one of the top factors in determining dividend policies. Specifically, established companies with stable, predictable income streams are more likely to pay dividends than companies with growing or volatile income.\\
	Newer and rapidly growing companies rarely pay dividends, as they prefer to invest their profits back into the company to fuel even more future growth. And, companies with unstable revenue streams often choose not to pay dividends, or pay small dividends in order to make sure the payout will be sustainable.\\
	It looks terrible to investors when companies are forced to suspend or reduce dividend payments, so most like to err on the side of caution when deciding to implement a new dividend, waiting for several years of stable profits before doing so.	
	
	\item  \textbf{Profitability.}
	Another factor that can influence management's dividend policies is the potential for better returns through capital reinvestment. In other words, if a company feels that it would be in the best interest of its shareholders to use its profits for other business activities besides paying dividends, it could choose not to pay – even if its revenues are stable and predictable.\\
	One great example of this is Warren Buffett's Berkshire Hathaway, which has never paid a dividend. Instead, Buffett feels that reinvesting the company's profits is a far better idea -- and he's been right. Berkshire has produced phenomenal returns for decades, and a big reason was the compounding effect of reinvesting its profits instead of paying them out.
	
	\item \textbf{Taxes.}
	Dividends are effectively taxed twice -- once at the corporate level, and again when they are paid out to shareholders.\\
	Because of this, many companies (and their investors) feel that other methods of returning capital, such as share repurchases, are a better way to go. Repurchasing shares has the same net effect as a dividend payment -- the intrinsic value of the company's shares increases as the share count drops. However, this can allow investors who like to reinvest their dividends to do so without having to worry about dividend taxes.
	
	\item \textbf{Legal requirements.} 
	It's also worth noting that some companies have no choice but to pay dividends. For example, real estate investment trusts receive some pretty nice tax benefits, but they are legally required to pay at least 90\% of their income to shareholders.\\
	On the other hand, some companies need to obtain approval before paying or increasing their dividends. Since the financial crisis, many banks need to submit capital plans for regulatory approval for any plans to boost their payouts.
	
	\item \textbf{Economic conditions.}
	Finally, another major factor that influences dividend policies is the market environment. If a certain sector is having trouble and anticipates profits falling, it's common for companies to get quite defensive when it comes to their dividends.\\	
	This can be seen currently in the energy sector, where low oil prices have wreaked havoc on many companies' profitability, which has led to several major companies slashing their dividends recently\footnote{JD.com, Inc. annual report 2016, Page 130}.
\end{itemize}

As for JD.com, inc, it stated that the form, frequency and amount of dividend payment will depend upon its future operations and earnings, capital requirements and surplus, general financial condition, contractual restrictions and other factors that the board of directors may deem.


\section{Dividend Policy performed by JD.com}
JD.com, inc does not reflect an active attitudes towards dividend payment. So far JD.com have not declared or paid any dividends on ordinary shares, nor do JD.com have any present plan to pay any cash dividends on ordinary shares in the foreseeable future.





\chapter{Current Assets and Current Liabilities}

\begin{center}
	
Current Assets (In thousands of RMB\textyen)

	\begin{tabular}{lrr}
		\toprule
		Fiscal year&2015&2016\\
		\midrule
		Short-term borrowings&17,863,868&19,771,695\\
		Nonrecourse securitization debt&2,114,913&4,391,955\\
		Accounts payable&2,780,482&4,391,955\\
		Advance from customers&-&723,449\\
		Deferred revenues&8,193,665&17,464,408\\
		Taxes payable&927,177&1,423,736\\
		Amount due to related parties&20,539,543&28,909,438\\
		Accrued expenses and other current liabilities&3,698,488&12,697,915\\
		Total current liabilities&-&10,766,920\\
		\bottomrule
	\end{tabular}
\end{center}

\begin{center}
	
Current Liabilities (In thousands of RMB\textyen)

\begin{tabular}{lrr}
	\toprule
	Fiscal year&2015&2016\\
	\midrule
	Cash and cash equivalents&17,863,868&19,771,695\\
	Restricted cash&2,114,913&4,391,955\\
	Short-term investments&2,780,482&4,391,955\\
	Investment securities&-&723,449\\
	Accounts receivable, net&8,193,665&17,464,408\\
	Advance to suppliers&927,177&1,423,736\\
	Inventories, net&20,539,543&28,909,438\\
	Loan receivables, net&3,698,488&12,697,915\\
	Other investments&-&10,766,920\\
	Prepayments and other current assets&1,486,441&2,198,906\\
	Amount due from related parties&863,516&1,410,050\\
	\bottomrule
\end{tabular}
\end{center}

\chapter{Example Chapter}

This is my text with an example Figure~\ref{fig:example} and example
citation~\cite{StrunkWhite} or \textcite{Bringhurst1993}. And there is another
\enquote{citation} which is located at the bottom\footcite{tagstore}.

\myfig{TU_Graz_Logo}%% filename in figures folder
  {width=0.1\textwidth,height=0.1\textheight}%% maximum width/height, aspect ratio will be kept
  {Example figure.}%% caption
  {}%% optional (short) caption for table of figures
  {fig:example}%% label

Now you are able to write your own document. Always keep in mind: it's
the \emph{content} that matters, not the form. But good typography is
able to deliver the content much better than information set with bad
typography. This template allows you to focus on writing good content
while the form is done by the template definitions.


%% vim:foldmethod=expr
%% vim:fde=getline(v\:lnum)=~'^%%%%\ .\\+'?'>1'\:'='
%%% Local Variables: 
%%% mode: latex
%%% mode: auto-fill
%%% mode: flyspell
%%% eval: (ispell-change-dictionary "en_US")
%%% TeX-master: "main"
%%% End: 
